\documentclass{article}

% pacotes para a lingua Portuguesa
\usepackage[portuguese]{babel}
\usepackage[utf8]{inputenc}

% pacote para inserção de imagens
\usepackage{graphicx}

% outros pacotes
\usepackage{url}


\title{Aplicações da tecnologia blockchain}
\author{Ruben Farinha, Rodrigo Isaque}
\date{\today}


\begin{document}
\maketitle

% ==== Primeira Secção ====  
\section{Tema}
A blockchain é uma tecnologia inovadora que está a revolucionar diversos setores, como o financeiro, saúde, logística. Essencialmente, é um livro de registos digital distribuído que regista transações de maneira segura e transparente, sem a necessidade de intermediários.

Uma das principais características da blockchain é a descentralização. Ao contrário de sistemas tradicionais, não há uma autoridade central que controle a rede. Em vez disso, a rede é mantida pelos próprios usuários, que validam as transações e garantem a segurança por meio de algoritmos criptográficos. Além disso, a blockchain é imutável, o que significa que, uma vez que uma transação é registrada na rede, ela não pode ser alterada ou apagada.

A blockchain tem diversas aplicações. Uma das mais conhecidas é a criptomoeda, como a Bitcoin, que utiliza a blockchain para registrar transações de forma segura e descentralizada. Além disso, a blockchain pode ser usada para criar contratos inteligentes, que são acordos digitais que se executam automaticamente quando as condições pré-determinadas são cumpridas. Isso pode ser útil em diversas áreas, desde a financeira até a logística.

Outra aplicação promissora da tecnologia blockchain é a criação de registos médicos eletrônicos descentralizados e seguros. Isso pode melhorar a eficiência dos serviços de saúde, permitindo o acesso aos registos médicos dos pacientes em qualquer lugar do mundo e garantindo a privacidade dos dados dos pacientes. Além disso, a blockchain pode ser usada para melhorar a gestão das cadeias de abastecimentos, permitindo o rastreamento dos produtos desde a origem até o destino final, garantindo a autenticidade e a qualidade.

No entanto, ainda existem desafios a serem superados. Um dos principais é a escalabilidade, ou seja, a capacidade da blockchain lidar com um grande número de transações. Outro desafio é a interoperabilidade, ou seja, a capacidade de diferentes blockchains comunicarem entre si.

Para superar esses desafios, várias soluções estão sendo desenvolvidas. Por exemplo, as sidechains são blockchains secundárias que se comunicam com a blockchain principal, permitindo a escalabilidade da rede. Além disso, soluções de interoperabilidade, como a Polkadot, estão sendo desenvolvidas para permitir a comunicação entre diferentes blockchains.

Em resumo, a tecnologia blockchain é uma oportunidade única de inovação e disrupção em diversas áreas. Embora ainda haja desafios a serem superados, a blockchain tem sido desenvolvida de maneira acelerada, com a criação de novas soluções que permitem sua escalabilidade e interoperabilidade. Como resultado, é uma área de pesquisa e desenvolvimento com um grande potencial de crescimento futuro.

\section{Bibliography}
    
    \cite{MasteringBitcoin}
    \cite{TheTruthMachine}
    \cite{WhatisBitcoinTechnology}
    \cite{BlockchainTechnology}
    \cite{Bitcoincashsystem}
    \cite{BitcoinandCryptocurrency}
    \cite{BlockchainNewEconomy}
    \cite{BlockchainRevolution}
    \cite{NextGenerationSmartContract}
    
        \bibliographystyle{plainurl}
        \bibliography{Bibliography}
\end{document}